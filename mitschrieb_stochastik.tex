\documentclass[a4paper,11pt,notitlepage]{report}

\usepackage{graphicx}
\usepackage[utf8]{inputenc}
\usepackage[T1]{fontenc}
\usepackage[ngerman]{babel}
\usepackage{bibgerm}
\usepackage{amsmath,amssymb,amsthm}
\usepackage{color}
\usepackage{enumerate}
\usepackage{tabularx}
\usepackage{subfig}
\usepackage{fancyhdr}
\usepackage[pdftex,pdfpagelabels,colorlinks,backref,pagebackref]{hyperref}
\usepackage{tikz} % SELBST HINZUGEFÜGT
% == Set the heading style ===================================================
\setlength{\headheight}{14pt}
\pagestyle{fancyplain}
\renewcommand{\chaptermark}[1]{\markboth{#1}{}}
\renewcommand{\sectionmark}[1]{\markright{\thesection\ #1}}
\lhead[\fancyplain{}{\thepage}]{\fancyplain{}{\rightmark}}
\rhead[\fancyplain{}{\leftmark}]{\fancyplain{}{\thepage}}
\cfoot{}
\renewcommand{\headrulewidth}{0.4pt}
% ============================================================================

% == Set correct values for fitting floats ===================================
\tolerance=2000
\emergencystretch=10pt

\setcounter{topnumber}{3}
\setcounter{totalnumber}{5}
\setcounter{bottomnumber}{2}

% To make those darn floats fit where they should
\setcounter{totalnumber}{9}
\setcounter{topnumber}{9}
\setcounter{bottomnumber}{9}
\renewcommand{\textfraction}{0.00}
\renewcommand{\topfraction}{1.0}
\renewcommand{\bottomfraction}{1.0}
% ============================================================================

% == German definitions for theorems etc. ==================================== 
\newtheorem{definition}{Definition}[chapter]
\newtheorem{theorem}{Satz}[chapter]
\newtheorem{lemma}{Lemma}[chapter]
\newtheorem{proposition}{Proposition}[chapter]
\newtheorem{corollary}{Korollar}[chapter]
\newtheorem{observation}{Beobachtung}[chapter]
\newtheorem{fact}{Fakt}[chapter]
\newtheorem{remark}{Bemerkung}[chapter]
\newtheorem{example}{Beispiel}[chapter]
% ============================================================================

% == Abkürzungen für die reellen, natürlichen, ganzen,... Zahlen =============
\newcommand{\R}{{\ensuremath{\mathbb{R}}}}
\newcommand{\N}{{\ensuremath{\mathbb{N}}}}
\newcommand{\Z}{{\ensuremath{\mathbb{Z}}}}
\newcommand{\C}{{\ensuremath{\mathbb{C}}}}
\newcommand{\Q}{{\ensuremath{\mathbb{Q}}}}
\newcommand{\F}{{\ensuremath{\mathbb{F}}}}
\newcommand{\Prim}{{\ensuremath{\mathbb{P}}}}
% ============================================================================

% == Makros für Autorenname und -adresse =====================================
\newcommand{\myaddress}[6]{%
  \parbox{\textwidth}{\textbf{\large #1}\\
    #2\\ #3\\ #4\\ 
    \ifthenelse{\equal{#5}{}}{}{Email: \href{mailto:#5}{\texttt{#5}}\\}
    \ifthenelse{\equal{#6}{}}{}{WWW: \href{#6}{\path|#6|}\\}
  } 
}

\newcommand{\myauthor}[1]{%
  \addtocontents{toc}{\protect\hspace{3.35ex}%
  \textsl{#1}\par}\vspace{-4ex}\quad\hfill\textsl{\Large #1}\vspace{8ex}}

\newcommand{\myname}[1]{\Large #1}

\title{\textbf{{Einführung in die Stochastik - Mitschrieb} \\[5ex] 
    {\Large Vorlesung im Wintersemester 2011/2012\\[5ex]}}}

%%%%%%%%%%%%%%%%%%%%%%%%%%%%%%%%%%%%%%%%%%%%%%%%%%
% Tragen Sie in der folg. Zeile Ihren Namen ein: %
%%%%%%%%%%%%%%%%%%%%%%%%%%%%%%%%%%%%%%%%%%%%%%%%%%
\author{\myname{Sarah Lutteropp}}

\begin{document}
\shorthandoff{"}
\maketitle
\setcounter{tocdepth}{1}
\tableofcontents

\section*{Vorwort}
Dies ist ein Mitschrieb der Vorlesung “Einführung in die Stochastik” vom Wintersemester 2011/2012 am Karlsruher Institut für Technologie, die von Herrn Prof. Dr. Günther Last gehalten wird.

\chapter{Deskriptive Statistik}
\section{Der Grundraum}
$\emptyset \neq \Omega$ = Grundraum (Grundgesamtheit, Merkmalsraum, Stichprobenraum)
Annahme: $\Omega$ ist diskret(endlich oder abzählbar unendlich) (Häufig $\Omega \subseteq \R$)

\section{Absolute und relative Häufigkeit}
$x_1, \ldots, x_n \in \Omega$ ("Daten") \newline
$h(\omega) = \text{card}\left\{j\in\{1, \ldots, n\} \colon x_j = \omega\right\}, \omega \in \Omega$, absolute Häufigkeit von $\omega$

\paragraph{Bemerkung}
$\sum\limits_{\omega \in \Omega}{h(\omega)} = n$

\paragraph{Definition}
$\frac{1}{n} h(\omega)$ = relative Häufigkeit von $\omega$ \newline
$h(A)=\text{card}\left\{j\in\{1,\ldots,n\}\colon x_j \in A\right\}, A \subset \Omega$ = absolute Häufigkeit von A, $\frac{1}{n} h(A)$ = relative Häufigkeit von A

\section{Histogramm}
$x_1, \ldots, x_n \in \R, b_1 < b_2 < \ldots < b_s$ mit $b_1 \leq \min\limits_{1 \leq i \leq n}{x_i}, b_s > \max\limits_{1 \leq i \leq n}{x_i}$
\newline
TODO: BILD
\newline
$d_j(b_{j+1}-b_j)=h([b_j,b_{j+1})) = \text{card} \left\{i\in\{1,\ldots,n\}\colon b_j \leq x_i < b_{j+1}\right\}$

\section{Lagemaße}
\paragraph{Definition}
Ein \textbf{Lagemaß} ist eine Abbildung $l \colon \R^n \rightarrow \R$ mit $$l(x_1+a,\ldots,x_n+a) = l(x_1,\ldots,x_n)+a$$ "Verschiebungskovarianz".
$x_1,\ldots,x_n,a \in \R$

\subsection{Arithmetisches Mittel}
$x_1,\ldots,x_n \in \R, \bar{x} := \frac{1}{n} \sum\limits_{j=1}^{n}{x_j}$ "Schwerpunkt der Daten"

\paragraph{Fakt}
$\sum\limits_{j=1}^{n}{(x_i - t)^2} \overset{t}{\rightarrow} \text{Min}$
\newline
Lösung: $t = \bar{x}$
\newline
"Prinzip der kleinsten Quadrate"

\paragraph{Beweis}
$\frac{1}{n} \sum\limits_{j=1}^{n}{(x_j - t)^2} = t^2 - 2\bar{x}t + \frac{1}{n} \sum\limits_{j=1}^{n}{x_j^2} = (t - \bar{x})^2 + \frac{1}{n} \sum\limits_{j=1}^{n}{x_j^2 - (\bar{x})^2}$

\subsection{Median, Quantile}
$x_1,\ldots,x_n \in \R \Rightarrow x_{(1)} \leq x_{(2)} \leq \ldots \leq x_{(n)}$ geordnete Stichprobe

\paragraph{Definition}

$$x_{1/2}:= \begin{cases}
	x_{(\frac{n+1}{2})} & \text{, falls } n \text{ ungerade} \\
	\frac{1}{2}(x_{(\frac{n}{2})} + x_{(\frac{n}{2}+1)}) & \text{, falls } n \text{ gerade}
\end{cases}
$$ 
heißt \textbf{Median} von $x_1,\ldots,x_n$.

\paragraph{Fakt}
$\sum\limits_{j=1}^{n}{|x_j - x_{1/2}|} = \min\limits_{t}{\sum\limits_{j=1}^{n}{|x_j - t|}}$ Übungsaufgabe

\paragraph{Bemerkung}
Der Median ist "robust" gegenüber "Ausreißern".
Ist etwa $x_1 = \ldots = x_9 = 1$ und $x_{10} = 1000 (n=10)$, so gilt $\bar{x} = 100,9  , x_{1/2} = 1$

\paragraph{Definition}
Für 0 < p < 1 heißt
$$
x_p := \begin{cases}
	x_{(\lfloor n \cdot p + 1 \rfloor )} & \text{, falls } n \cdot p \notin \N \\
	\frac{1}{2}(x_{(n \cdot p)} + x_{(n \cdot p + 1)}) & \text{, falls } n \cdot p \in \N
\end{cases}
$$
\textbf{p-Quantil} von $x_1, \ldots, x_n$.

\paragraph{Interpretation}
Mindestens $p \cdot 100 \%$ der Daten liegen links von $x_p$ und mindestens $(1-p) \cdot 100 \%$ liegen rechts von $x_p$. \newline
$x_{1/4}=$ unteres Quartil, $x_{3/4}=$ oberes Quartil

\section{Streuungsmaße}
\paragraph{Definition}
Eine Abbildung $\sigma \colon \R^n \rightarrow \R$ mit $$\sigma(x_1+a,\ldots,x_n+a) = \sigma(x_1,\ldots,x_n)\text{ (Translationsinvarianz)}$$ heißt \textbf{Streuungsmaß}.

\subsection{Empirische Varianz}
$s^2 := \frac{1}{n-1} \sum\limits_{j=1}^{n}{(x_j - \bar{x})^2}$ = \textbf{empirische Varianz} von $x_1,\ldots,x_n$

\subsection{Empirische Standardabweichung}
$s := + \sqrt{s^2}$ = \textbf{empirische Standardabweichung} von $x_1,\ldots,x_n$

\subsection{Spannweite}
$x_{(n)} - x_{(1)}$ = \textbf{Spannweite} von $x_1,\ldots,x_n$

\subsection{Quartilsabstand}
$x_{(3/4)} - x_{(1/4)}$ = \textbf{Quartilsabstand} von $x_1,\ldots,x_n$

\section{Empirischer Korrelationskoeffizient}

$(x_1,y_1), \ldots, (x_n,y_n) \in \R^2$
TODO: BILD

Gesucht: Gerade $y = a + b \cdot x$ so, dass
$$(*) \sum\limits_{j=1}^{n}{(y_j - a - b x_j)^2} \overset{a,b}\rightarrow \text{Min}$$

\paragraph{Definition}
$\sigma_{x}^2 = \frac{1}{n}\sum\limits_{j=1}^{n}{(x_j - \bar{x})^2}$
$\sigma_{y}^2 = \frac{1}{n}\sum\limits_{j=1}^{n}{(y_j - \bar{y})^2}$

$\sigma_{xy} = \frac{1}{n}\sum\limits_{j=1}^{n}{(x_j - \bar{x})(y_j - \bar{y})}$ \textbf{empirische Kovarianz} $\sigma_x^2 > 0, \sigma_y^2 >0.$

Lösung von (*):
$b^* = \frac{\sigma_{xy}}{\sigma_{x^2}}, a^*= \bar{y} - b^* \cdot \bar{x}$


$\min\limits_{a,b} {\sum\limits_{j=1}^{n}{(y_j - a - b x_j)^2}} \stackrel{!}{=} \min\limits_{b}{\sum\limits_{j=1}^{n}{(y_i - \bar{y} - b (x_j - \bar{x}))^2}}=\ldots$

"lineare Regression"
\newline

Einsetzen von $a^*$ und $b^*$ in die Zielfunktion:
$$0 \leq \sum\limits_{j=1}^{n}{(y_j - a^* - b^* x_j)^2} = \ldots = n \sigma_y^2 (1-(\frac{\sigma_{xy}}{\sigma_x \sigma_y})^2)$$

\paragraph{Definition}
$r_{xy}:= \frac{\sigma_{xy}}{\sigma_x \sigma_y}$ heißt \textbf{empirischer Korrelationskoeffizient} (\emph{Pearson}).

\paragraph{Folgerung}
$|r_{xy}|\leq 1$
\newline
Es gilt $r_{xy} = \pm 1 \Leftrightarrow$ Punktewolke liegt exakt auf der Geraden $y=a^*+b^*x$.
Dabei ist $b^* > 0$, falls $r_{xy} = 1$ und $b^* < 0$, falls $r_{xy} = -1$.
\newline 
\emph{Dieser empirische Korrelationskoeffizient ist ein Maß für die (affin) lineare Abhängigkeit zwischen den $x_j$ und den $y_j$.}

\chapter{Ereignisse und Zufallsvariablen}

\section{Definition}
Gegeben sei eine \underline{Grundmenge} $\Omega$. Die Elemente von $\Omega$ heißen \textbf{Elementarereignisse}. Teilmengen von $\Omega$ heißen \textbf{Ereignisse}. (Idee: $\omega \in \Omega$ ist Ausgang eines zufälligen Versuchs.)

\paragraph{Interpretation}
Ein Ereignis $A \subset \Omega$ "tritt ein", wenn $\omega \in A$.

\section{Beispiele}
\begin{itemize}
	\item (i) (Münzwurf) \newline
		$\Omega = \{0,1\} (\text{oder } \Omega = \{W,Z\})$
	\item (ii) (m Münzwürfe) \newline
		$\Omega = \{0,1\}^m (A = \{\omega = (\omega_1, \ldots, \omega_m) : \sum\limits_{j=1}^{m}{\omega_j} \geq k\} \text{ Ereignis })$
	\item (iii) Werfen von 2 Würfeln \newline
		$\Omega = \{1, \ldots, 6\}^2$
	\item (iv) Brownsche Bewegung \newline
		(TODO: BILD) Bewegung eines Blütenpollens in einer Flüssigkeit
		\newline
		$\Rightarrow$ Zukunftsmusik
		\newline
		$\Omega = C([0,1], \R^2)$
\end{itemize}

\section{Bemerkung (Mengentheoretische Operationen)}
Seien $A, B, A_1, A_2, \ldots \subset \Omega$.
\newline
$A \cap B = \{\omega \in \Omega : \omega \in A \text{ und } \omega \in B\} \mathop{\hat{=}}  \text{"A und B treten ein"}$
\newline
$A \cup B \mathop{\hat{=}}  \text{"A oder B treten ein"}$
\newline
$\bar{A} \equiv A^c := \Omega \backslash A = \{\omega \in \Omega : \omega \notin A \} \mathop{\hat{=}} \text{"A tritt nicht ein"}$
\newline
$A \backslash B = A \cap B^c \mathop{\hat{=}} \text{"A tritt ein, aber nicht B"}$
\newline
$A \subset B \mathop{\hat{=}} \text{"wenn A, dann B"}$
\newline
$\emptyset \mathop{\hat{=}} \text{"unmögliches Ereignis"}$
\newline
$\Omega \mathop{\hat{=}} \text{"sicheres Ereignis"}$
\newline
\paragraph{Abkürzung} $AB = A \cap B$

\section{Definition}
Eine Abbildung $X \colon \Omega \rightarrow \R$ heißt (reelle) \textbf{Zufallsvariable}. Für $\omega \in \Omega$ heißt $X(\omega)$ \textbf{Realisierung} der Zufallsvariable zu $\omega$.

\paragraph{Idee} 
Mit $\omega \in \Omega$ bekommt auch $X(\omega)$ einen zufälligen Charakter.

\paragraph{Definition}
$X^{-1} \colon \mathcal{P}(\R) \rightarrow \mathcal{P}(\Omega) = \{A \colon A \in \Omega\}$ ist definiert durch 
$$X^{-1}(A) = \{\omega \in \Omega \colon X(\omega) \in A\} \text{ ("Urbild von A unter X")}$$

\paragraph{Bemerkung}
\begin{itemize}
\item $X^{-1}(A \cap B) = X^{-1}(A) \cap X^{-1}(B), A,B \subset \R$
\item $X^{-1}(A \cup B) = X^{-1}(A) \cup X^{-1}(B)$
\item $X^{-1}(\bigcup\limits_{j=1}^{\infty}{A_j}) = \bigcup\limits_{j=1}^{\infty}{X^{-1}(A_j)}$
\item $X^{-1}(\bigcap\limits_{j=1}^{\infty}{A_j}) = \bigcap\limits_{j=1}^{\infty}{X^{-1}(A_j)}$
\end{itemize}

\paragraph{Vereinbarung}
Es sei $X$ eine Zufallsvariable und $t \in \R$. Wir setzen
\begin{itemize}
 \item $\{X=t\} := \{\omega \colon X(\omega) = t \} (= X^{-1}(t))$
 \item $\{X \geq t\} := \{\omega \colon X(\omega) \geq t \}$
\end{itemize}

\section{Definition}
Sind $X,Y$ Zufallsvariablen, so definiert man
\begin{itemize}
	\item $(X+Y)(\omega) = X(\omega) + Y (\omega)$
	\item $(X-Y)(\omega) = X(\omega) - Y (\omega)$
	\item $(X \cdot Y)(\omega) = X(\omega) \cdot Y (\omega)$
\end{itemize}
$\omega \in \Omega, \text{neue Zufallsvariablen } X+Y, X-Y, X \cdot Y$
\newline
analog für $a \in \R$
\begin{itemize}
	\item $a X(\omega) = a \cdot (X(\omega))$
	\item $\min(X,Y) = (X \wedge Y)(\omega):= \min \{X(\omega), Y(\omega)\} \ldots$
\end{itemize}

\section{Definition}
Sei $A \subset \Omega$. Die Funktion $1_A \colon \Omega \rightarrow \R$ ist definiert durch
$$1_A(\omega) = \begin{cases} 1 & \text{, falls } \omega \in A \\
							  0 & \text{, falls } \omega \notin A						
\end{cases}
$$
und heißt \textbf{Indikatorfunktion} von $A$.

\section{Bemerkungen (Rechenregeln für Indikatorfunktionen)}

\begin{itemize}
	\item $1_{\emptyset} \equiv 0$
	\item $1_{\Omega} \equiv 1$
	\item $(1_A)^2 = 1_A$
	\item $1_{A^c} = 1 - 1_A$
	\item $1_{A \cap B} = 1_A \cdot 1_B$
	\item $1_{A \cup B} = 1_A + 1_B - 1_{A \cap B}$
	\item $A \subset B \Leftrightarrow 1_A \leq 1_B$
	\item $1_{A \triangle B} = |1_A - 1_B|$
\end{itemize}

\section{Definition}
Seien $A_1, \ldots, A_n \subset \Omega$. Die Zufallsvariable 
$$X := \sum\limits_{j=1}^{n}{1_{A_j}}$$ heißt \textbf{Zählvariable} oder \textbf{Indikatorsumme}.

\paragraph{Bemerkung}
\begin{itemize}
	\item $\{X=0\} = \{\omega \colon X(\omega) = 0 \} = A_{1}^c \cap \ldots A_{n}^c$
	\item $\{X = n \} = A_1 \cap \ldots \cap A_n$
	\item $\{X = k\} = $ "genau k der Ereignisse $A_1, \ldots, A_n$ treten ein" = $\bigcup\limits_{T \subset \{1, \ldots, n\}, |T|=k}{\bigl (\bigcap\limits_{j \in T}{A_j} \cap \bigcap\limits_{j \notin T}{A_{j}^c} \bigr )}$ \newline
$(T \subset \{1, \ldots, n\}, |T| = \text{card } T = k)$
\end{itemize}

\chapter{Diskrete Wahrscheinlichkeitsräume}

\section{Motivation}
Zufallsexperiment mit Ausgängen in $\Omega$
\newline
n-malige, `unabhängige' Wiederholung
\newline
$\Rightarrow$ Ergebnis $(a_1, \ldots, a_n) \in \Omega^n$
\newline
$r_n(A):= \frac{1}{n} \sum\limits_{j=1}^{n}{1_A(a_j)}, A \subset \Omega$ relative Häufigkeit von $A$
\newline
$0 \leq r_n(A) \leq 1, r_n(\emptyset) = 0, r_n(\Omega) = 1$
\newline
$r_n(A \cup B) = r_n(A) + r_n(B), A \cap B = \emptyset$
\newline
empirisches Gesetz über Stabilisierung relativer Häufigkeiten:
\newline
$r_n(A) \underset{n \rightarrow \infty}{\leadsto} ? $

\section{Definition}
	Ein Paar $(\Omega, \mathbb{P})$ bestehend aus einer diskreten Menge $\Omega \neq \emptyset$ und einer Funktion $\mathbb{P} \colon \mathcal{P} \rightarrow \R$ heißt \textbf{diskreter Wahrscheinlichkeitsraum}, falls:
	\begin{itemize}
		\item (P1) $\mathbb{P}(A) \geq 0, A \subset \Omega$
		\item (P2) $\mathbb{P}(\Omega) = 1$
		\item (P3) $\mathbb{P}(\bigcup\limits_{j = 1}^{\infty}{A_j}) = \sum\limits_{j=1}^{\infty}{\mathbb{P}(A_j)}, A_i \cap A_j = \emptyset, i \neq j$
		\newline
		Diese Eigenschaft heißt $\sigma$-Additivität.
	\end{itemize}
	Man nennt $\mathbb{P}$ \textbf{Wahrscheinlichkeitsmaß (auf $\Omega$)} (oder Wahrscheinlichkeitsverteilung) und $\mathbb{P}(A)$ heißt \textbf{Wahrscheinlichkeit von A}.
	
\section{Folgerung}
\begin{itemize}
	\item a) $\Prim(\emptyset) = 0$
	\item b) $\Prim(\bigcup\limits_{j=1}^{n}{A_j}) = \sum \limits_{j=1}^{n}{\Prim(A_j)}, A_i \cap A_j = \emptyset, i \neq j$
	\item c) $0 \leq \Prim(A) \leq 1, A \subset \Omega$
	\item d) $\Prim(A \cup B) = \Prim(A) + \Prim(B) - \Prim(A \cap B), A,B \subset \Omega$
	\item e) $A \subset B \Rightarrow \Prim(A) \leq \Prim(B)$ (Monotonie)
	\item f) $\Prim(A^c) = 1 - \Prim(A)$ (Komplementärwahrscheinlichkeit)
	\item g) $\Prim(\bigcup\limits_{j=1}^{\infty}{A_j}) \leq \sum \limits_{j=1}^{\infty}{A_j}$ (Subadditivität)
	\item h) $A_n \subset A_{n+1}, n \in \N \Rightarrow \Prim(\bigcup\limits_{n=1}^{\infty}{A_n}) = \lim \limits_{n \rightarrow \infty}{\Prim(A_n)}$ (Stetigkeit von unten)
	\item i) $A_n \supset A_{n+1}, n \in \N \Rightarrow \Prim(\bigcap\limits_{n=1}^{\infty}{A_n}) = \lim \limits_{n \rightarrow \infty}{\Prim(A_n)}$ (Stetigkeit von oben)
\end{itemize}

\paragraph{Beweis}
$\bullet$ a): $A_j = \emptyset, j \in \N$ (P3) \footnote{$\Prim(\emptyset) = \Prim(\emptyset \cup \emptyset) = \Prim(\emptyset) + \Prim(\emptyset) = 2 \cdot \Prim(\emptyset)$} $\Prim(\emptyset) = 0$.
\newline
$\bullet$ b): $A_{n+1} = A_{n+2} = \ldots = \emptyset$ in P3!
\newline
$\bullet$ c) + f): Für $A \subset \Omega$ gilt nach b) (für n = 2): \newline
	$1 = \Prim(\Omega) = \Prim(A \cup A^c) \overset{(b)}{=} \Prim(A) + \Prim(A^c)$
\newline
$\bullet$ d): Nach b) gilt $\Prim(A) = \Prim(A \backslash B) + \Prim(A \cap B)$, $\Prim(B) = \Prim(B \backslash A) + \Prim(A \cap B)$ und somit $\Prim(A) + \Prim(B) - \Prim(A \cap B) = \Prim(A \backslash B) + \Prim (B \backslash A) + \Prim(A \cap B) \overset{(b)}{=} \Prim(A \cup B)$
\newline
$\bullet$ e): Wegen $B = A \cup (B \backslash A)$ folgt\footnote{(aus der Additivität)}
	$\Prim(B) = \Prim(A) + \Prim(B \backslash A) \geq \Prim(A)$
\newline
$\bullet$ g): $B_1 := A_1, B_2 := A_2 \backslash A_1, \ldots , B_n := A_n \backslash (\bigcup \limits_{j=1}^{n-1}{A_j}), n \geq 2$.
\newline
Dann gilt $B_n \subset A_n$ und $\bigcup\limits_{j=1}^{n}{B_j} = \bigcup\limits_{j=1}^{n}{A_j}$ sowie $B_i \cap B_j = \emptyset, i \neq j.$
\newline
Es folgt aus (P3):
\newline
$\Prim(\bigcup\limits_{j=1}^{\infty}{A_j}) \overset{!}{=} \Prim(\bigcup\limits_{j=1}^{n}{B_j}) \overset {(P3)} {=} \sum\limits_{j=1}^{n}{\Prim(B_j)} \overset {e)}{\leq} \sum\limits_{j=1}^{n}{\Prim(A_j)}$ ($\infty$ ist zugelassen)
\newline
$\bullet$ h) + i): Übungsaufgabe

\section{Satz}
Seien $A_1, \ldots, A_n \subset \Omega.$ Setze
$$S_k := \sum\limits_{1 \leq i_1 < \ldots < i_k \leq n}{\Prim(A_{i_1} \cap \ldots \cap A_{i_k})}$$
Dann gilt
\begin{itemize}
	\item a) $\Prim(\bigcup\limits_{j=1}^{n}{A_j}) = \sum\limits_{k=1}^{n}{(-1)^{k-1} S_k}$ `Siebformel'
	\item b) $\Prim(\bigcup\limits_{j=1}^{n}{A_j}) \leq \sum\limits_{k=1}^{2s+1}{(-1)^{k-1}S_k}, s = 0, \ldots, \lfloor \frac{n-1}{2} \rfloor$
	\newline
	$\Prim(\bigcup\limits_{j=1}^{n}{A_j}) \geq \sum\limits_{k=1}^{2s}{(-1)^{k-1}S_k}, s = 1, \ldots, \lfloor \frac{n}{2} \rfloor$
\end{itemize}

\paragraph{Beweisidee für Siebformel}
vollständige Induktion nach $n$:
\newline
$\text{\underline{n=2:} }\Prim(A_1 \cup A_2) \overset{(d)}{=} \Prim(A_1) + \Prim(A_2) - \Prim(A_1 \cap A_2) = S_1 - S_2$
\newline
$\text{\underline{n=3:} }\Prim(\underbrace{A_1 \cup A_2} \cup A_3) \overset {(d)} {=} \Prim(A_1 \cup A_2) + \Prim(A_3) - \Prim((A_1 \cup A_2) \cap A_3)$ \footnote{$(A_1 \cup A_2) \cap A_3 = (A_1 \cap A_3) \cup (A_2 \cap A_3)$}
\newline
  $\overset {(d)} {=} \Prim(A_1) + \Prim(A_2) - \Prim(A_1 \cap A_2) + \Prim(A_3) - \Prim(A_1 \cap A_3) - \Prim(A_2 \cap A_3) + \Prim(A_1 \cap A_2 \cap A_3) = S_1 - S_2 + S_3$
  
\section{Definition + Satz}
a) Sei $(\Omega, \Prim)$ diskreter Wahrscheinlichkeitsraum. Dann heißt $p \colon \Omega \rightarrow \R$ definiert durch $p(\omega) := \Prim(\{\omega\})$ \textbf{Wahrscheinlichkeitsfunktion} (von $\Prim$).
\newline
Es gilt $\Prim(A) = \sum\limits_{\omega \in A}{p(\omega)}, A\subset \Omega$.
\newline
b) Sind $\Omega$ diskret und $p \colon \Omega \rightarrow \R$ eine Abbildung mit $p(\omega)\geq 0$ und $\sum\limits_{\omega \in \Omega}{p(\omega)} = 1$, so erhält man vermöge $\Prim(A):= \sum\limits_{\omega \in A}{p(\omega)}$ einen diskreten Wahrscheinlichkeitsraum.

\paragraph{Beweis}
$\bullet$ a) $\sigma$-Additivität ($A = \bigcup\limits_{\omega \in A}{\{\omega\}}$)
\newline
$\bullet$ b) $\sigma$-Additivität: Großer Umordnungssatz! (Analysis)

\section{Definition}
$|\Omega| =: n < \infty$. Definiere $\Prim(A)= \frac{|A|}{n}.$
Dann heißt $(\Omega, \Prim)$ (ein diskreter Wahrscheinlichkeitsraum!) \textbf{Laplace-Raum}. Man nennt $\Prim$ \textbf{Gleichverteilung} auf $\Omega$. 
\newline
(`homogene Münze', `Würfeln', \ldots)

\paragraph{Definition}
Sei $\Omega \neq \emptyset$ beliebig!
$(\Omega, \Prim)$ diskreter Wahrscheinlichkeitsraum $\Leftrightarrow \exists$ abzählbare Menge $\Omega_0 \subset \Omega$, $\exists p \colon \Omega \rightarrow [0, \infty)$ mit $p(\omega=0)$ für alle $\omega \notin \Omega_0$, und $\sum\limits_{\omega \in \Omega_0}{p(\omega)} = 1$, und $\Prim(A) = \sum \limits_{\omega \in A \cap \Omega_0}{p(\omega)}, A \subset \Omega$.


\end{document}
