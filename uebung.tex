\documentclass[a4paper,11pt,notitlepage]{report}

\usepackage{graphicx}
\usepackage[utf8]{inputenc}
\usepackage[T1]{fontenc}
\usepackage[ngerman]{babel}
\usepackage{bibgerm}
\usepackage{amsmath,amssymb,amsthm}
\usepackage{color}
\usepackage{enumerate}
\usepackage{tabularx}
\usepackage{subfig}
\usepackage{fancyhdr}
\usepackage[pdftex,pdfpagelabels,colorlinks,backref,pagebackref]{hyperref}
\usepackage{tikz} % SELBST HINZUGEFÜGT
\usepackage{slashbox}
% == Set the heading style ===================================================
\setlength{\headheight}{14pt}
\pagestyle{fancyplain}
\renewcommand{\chaptermark}[1]{\markboth{#1}{}}
\renewcommand{\sectionmark}[1]{\markright{\thesection\ #1}}
\lhead[\fancyplain{}{\thepage}]{\fancyplain{}{\rightmark}}
\rhead[\fancyplain{}{\leftmark}]{\fancyplain{}{\thepage}}
\cfoot{}
\renewcommand{\headrulewidth}{0.4pt}
% ============================================================================

% == Set correct values for fitting floats ===================================
\tolerance=2000
\emergencystretch=10pt

\setcounter{topnumber}{3}
\setcounter{totalnumber}{5}
\setcounter{bottomnumber}{2}

% To make those darn floats fit where they should
\setcounter{totalnumber}{9}
\setcounter{topnumber}{9}
\setcounter{bottomnumber}{9}
\renewcommand{\textfraction}{0.00}
\renewcommand{\topfraction}{1.0}
\renewcommand{\bottomfraction}{1.0}
% ============================================================================

% == German definitions for theorems etc. ==================================== 
\newtheorem{definition}{Definition}[chapter]
\newtheorem{theorem}{Satz}[chapter]
\newtheorem{lemma}{Lemma}[chapter]
\newtheorem{proposition}{Proposition}[chapter]
\newtheorem{corollary}{Korollar}[chapter]
\newtheorem{observation}{Beobachtung}[chapter]
\newtheorem{fact}{Fakt}[chapter]
\newtheorem{remark}{Bemerkung}[chapter]
\newtheorem{example}{Beispiel}[chapter]
% ============================================================================

% == Abkürzungen für die reellen, natürlichen, ganzen,... Zahlen =============
\newcommand{\R}{{\ensuremath{\mathbb{R}}}}
\newcommand{\N}{{\ensuremath{\mathbb{N}}}}
\newcommand{\Z}{{\ensuremath{\mathbb{Z}}}}
\newcommand{\C}{{\ensuremath{\mathbb{C}}}}
\newcommand{\Q}{{\ensuremath{\mathbb{Q}}}}
\newcommand{\F}{{\ensuremath{\mathbb{F}}}}
\newcommand{\Prim}{{\ensuremath{\mathbb{P}}}}
\newcommand{\E}{{\ensuremath{\mathbb{E}}}}
% ============================================================================

% == Makros für Autorenname und -adresse =====================================
\newcommand{\myaddress}[6]{%
  \parbox{\textwidth}{\textbf{\large #1}\\
    #2\\ #3\\ #4\\ 
    \ifthenelse{\equal{#5}{}}{}{Email: \href{mailto:#5}{\texttt{#5}}\\}
    \ifthenelse{\equal{#6}{}}{}{WWW: \href{#6}{\path|#6|}\\}
  } 
}

\newcommand{\myauthor}[1]{%
  \addtocontents{toc}{\protect\hspace{3.35ex}%
  \textsl{#1}\par}\vspace{-4ex}\quad\hfill\textsl{\Large #1}\vspace{8ex}}

\newcommand{\myname}[1]{\Large #1}

\title{\textbf{{Einführung in die Stochastik - Mitschrieb} \\[5ex] 
    {\Large Vorlesung im Wintersemester 2011/2012\\[5ex]}}}

%%%%%%%%%%%%%%%%%%%%%%%%%%%%%%%%%%%%%%%%%%%%%%%%%%
% Tragen Sie in der folg. Zeile Ihren Namen ein: %
%%%%%%%%%%%%%%%%%%%%%%%%%%%%%%%%%%%%%%%%%%%%%%%%%%
\author{\myname{Sarah Lutteropp}}

\begin{document}
\shorthandoff{"}
\maketitle
\setcounter{tocdepth}{1}
\tableofcontents

\section*{Vorwort}
Dies ist ein Mitschrieb der Übung “Einführung in die Stochastik” vom Wintersemester 2011/2012 am Karlsruher Institut für Technologie.

\chapter{24.11.11}

\subsection{Aufgabe 16}
\begin{proof}
	weiterhin gilt
	$$B^{i_1, \ldots, i_k}=\bigcap\limits_{\mu = 1}^k{A_{i_\mu}} \cap \bigcap\limits_{\nu \notin\{i_1, \ldots, i_k\}}{A_\nu^c}$$
	$$\Rightarrow 1_{\{X=k\}} = \sum\limits_{1 \leq i_1 < \ldots < i_k \leq n}{1_{B^{i_1, \ldots, i_k}}}$$
	$$= \sum\limits_{1 \leq i_1 < \ldots < i_k \leq n}{\prod\limits_{\mu=1}^k{1_{A_{i_\mu}}} \prod\limits_{\nu \notin\{i_1, \ldots, i_k\}}{(1-1_{1_\nu})}}$$
	allgemein gilt für  Funktionen $f$, $g$:
	$$\prod\limits_{\nu}^m{(f_\nu + g_\nu)} = \sum\limits_{r=0}^m{\sum\limits_{1 \leq i_1 < \ldots < i_r \leq m}{\prod\limits_{\nu = 1}^r{g_\nu} \prod\limits_{\mu \notin \{i_1, \ldots, i_r\}}{f_\mu}}}$$
	(Beweis mit vollst. Induktion über $m$)
	\newline
	hier $f_\nu \equiv 1, g_\nu \equiv -1_{A_\nu}$
	$$\Rightarrow \prod\limits_{\nu \notin\{i_1, \ldots, i_k\}}{(1-1_{A_\nu})} = \prod\limits_{\nu = j_1, \ldots, j_{n-k}}{(1-1_{A_\nu})}$$
		(wobei $\{j_1, \ldots, j_{n-k}\} = \{1,\ldots,n\} \backslash \{i_1, \ldots, i_k\}$)
	$$= \prod\limits_{\nu=1}^{n-k}{(1-1_{A_\nu})}$$
	$$= \sum\limits_{r=0}^{n-k}{\sum\limits_{1 \leq i_1 < \ldots < i_r \leq n}{\underbrace{\prod\limits_{\nu = 1}^r{(-1_{A_\nu})}}_{= (-1)^r \prod\limits_{\nu = 1}^r{1_{A_\nu}}} \cdot 1}}$$
	$$\Rightarrow 1_{\{X=k\}} = \sum\limits_{r=0}^{n-k}{\sum\limits_{1 \leq i_1 < \ldots < i_k \leq n}{\sum\limits_{1 \leq i_1 < \ldots < i_r \leq n, i_1, \ldots, i_r \notin \{i_1, \ldots, i_k\}}{\prod\limits_{\mu = 1}^k}{1_{A_{i_\mu}}}\prod\limits_{\nu = 1}^r{1_{A_{i \nu}}}}}$$
	statt erst $k$, dann weitere $r$ Indizes zu wählen, wähle nun direkt $(k+r)$ Indizes $\leadsto {k+r \choose k}$ Möglichkeiten
	\newline
	(z.B. für die Wahl der Indizes $1,2,3$ gibt es für $k=1$ die Möglichkeiten
	$$i_1 = 1, \quad j_1,j_2=2,3$$
	$$i_1 = 2, \quad j_1,j_2 = 1,3$$
	$$i_1 = 3, \quad j_1, j_2 = 1,2)$$
	$$\Rightarrow 1_{\{X=k\}} = \sum\limits_{r=0}^{n-k}{(-1)^r(\sum\limits_{1 \leq i_1 < \ldots < i_{k+r} \leq n}{\prod\limits_{\mu = 1}^{k+r}{1_{A_{i_\mu}}) \cdot ({k+r \choose k})})}}$$
	$$\Rightarrow P(X=k)=E(1_{\{X=k\}})$$
	$$= \sum\limits_{r=0}^{n-k}{(-1)^r{k+r \choose k} \sum\limits_{1 \leq i_1 y \ldots < i_{k+r}\leq n}{P(A_{i_1}\cap \ldots \cap A_{i_{k+r}})}}$$
	$j=r+k:$
	$$= \underbrace{\sum\limits_{j=k}^n{(-1)^{j-k}{j \choose k}S_j}}_{{k+r \choose k}={k+r \choose r} = {j \choose j-k} = {j \choose k}}$$
\end{proof}

\subsection{Zusatz 1}
\paragraph{Modell}
Fächer von 1 bis $n$ nummeriert.
Es werden $k$ Kugeln auf diese Fächer verteilt (gleichwahrscheinlich)
$$X_{n,k} \hat{=} \text{ Anzahl der Fächer, die leer geblieben sind}$$
$$\Omega = \{ \omega = (\omega_1, \ldots, \omega_k) \colon 1 \leq \omega_i \leq n\} = \{1, \ldots, n\}^k$$
$$P \text{ Gleichverteilung auf } \Omega$$
$$A_j = \{\text{j-tes Fach leer geblieben} \}$$
$$\Rightarrow X_{n,k} = \sum\limits_{j=1}^n{1_{A_j}}$$
Sei nun $1 \leq i_1 < \ldots < i_r \leq n$.
$$\bigcap\limits_{\nu=1}^r{A_{i_\nu}} = \{ \omega \colon \omega_1, \ldots, \omega_k \in \{1, \ldots,n\} \backslash \{i_1, \ldots, i_k\}\}$$
$$\Rightarrow P(\bigcap\limits_{\nu = 1}^r{{A}_{i_\nu}}) = \frac{(n-r)^k}{n^k} = \left(1-\frac{r}{n}\right)^k$$
$$\Rightarrow P(X_{n,k}=1) = \sum\limits_{r=j}^n{(-1)^{r-j}{r \choose j}\underbrace{S_r}_{{n \choose r}(1- \frac{r}{n})^k}}$$

\subsection{Aufgabe 12}
$\Omega = Per_{n, \neq}^n$
$$A_j = \{(a_1, \ldots, a_n) \in \Omega \colon a_j \geq 1\}$$
$$E(\sum\limits_{j=2}^n{1_{A_j}}) = \sum\limits_{j=2}^n{\underbrace{E(1_{A_j})}_{=P(A_j)}}$$
Berechne für $j=1, \ldots, n$
$$P(A_j)$$
$$|A_j| = \underbrace{n-j+1}_{\text{Möglichkeiten, den j-ten Eintrag festzulegen}} \cdot \underbrace{|Per_{n-1, \neq}^{n-1}|}_{=(n-1)!}$$
$$|\Omega| = n! \Rightarrow P(A_j) = \frac{n-j+1}{n}$$
$$\Rightarrow E\left( \sum\limits_{j=2}^n{1_{A_j}} \right)= \frac{1}{n} \sum\limits_{j=2}^n{(n-j+1)}$$
$$= \frac{1}{n}(n-1)(n+1) + \frac{1}{n}\underbrace{\sum\limits_{j=2}^n{(-j)}}_{= - \sum\limits_{j=1}^n{j}+1 = -\frac{n(n+1)}{2}+1}$$
$$=\ldots=\frac{n-1}{2}$$

\subsection{Aufgabe 15}
r rote, s schwarze Kugeln
\newline
nach dem Ziehen die Kugel und $c \in \{-1,0,1,\ldots\}$ zusätzliche zurücklegen.
$$\Omega := \Omega_1 \times \Omega_2 = \{0,1\}^2$$
$$0 \hat{=} \text{ schwarz}$$
Startverteilung:
$$p_1 \colon \Omega_1 \rightarrow [0,1]:$$
$$p_1(0):=\frac{s}{r+s}, p_1(1)=\frac{r}{r+s}$$
Übergangsverteilung:
$$p_2(0|0) = \frac{s+c}{r+s+c}, p_2(1|0) = \frac{r}{r+s+c}$$
$$p_2(0|1)= \frac{s}{r+s+c}, p_2(1|1)=\frac{r+c}{r+s+c}$$
$$z.B. P\left((0,1)\right) = p_1(0) \cdot p_2(1|0)= \frac{s}{r+s}\frac{r}{r+s+c}$$
\end{document}
